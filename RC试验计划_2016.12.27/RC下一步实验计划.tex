\documentclass[11pt]{article}
\usepackage[margin=1in]{geometry}          
\usepackage{graphicx}
\usepackage{amsthm, amsmath, amssymb}
\usepackage{setspace}\onehalfspacing
\usepackage[loose,nice]{units} %replace "nice" by "ugly" for units in upright fractions
 \usepackage[UTF8, heading = false, scheme = plain]{ctex}
 \usepackage{hyperref}
 \usepackage{color}
\usepackage[normalem]{ulem}
\usepackage{url}
\usepackage[dvipsnames]{xcolor}


\DeclareUrlCommand\ULurl{%
  \renewcommand\UrlFont{\ttfamily\color{blue}}%
  \renewcommand\UrlLeft{\uline\bgroup}%
  \renewcommand\UrlRight{\egroup}}
 
\title{阅读理解下一步实验计划}
\author{徐俊}
\date{2016年12月27日}
 
\begin{document}
\maketitle

\section{问题}
CNN数据集中,为了增加难度同时要求模型必须依赖于“上下文信息”获取答案,而将所有的数据中的实体(潜在答案)全部替换为@entity,一篇文章中同一个实体使用同一个@entity编号但是不同文章中的同一@entity编号却可能表示不同的实体。

目前的Reader均将@entity编号作为一个个独立的“词”来处理的,它们拥有自己的embedding,在softmax输出层(如果有的话)也是被当做单独的“词”来处理的。这样一来就有问题,在NN的模型中,embedding和softmax输出层的权重向量均对应一个独立的词语,而@entity符号却并不具备“独立”这个属性,因为同一个@entity符号在不同文章中代表不同的实体,而实体才是具备“独立”属性的”词“。

数据集本意是要求model仅仅依赖于上下文信息来做判断,@entity仅仅作为标示符出现而不含有任何语义。

但是实际操作过程中如Attention Reader中,最终的预测环节(softmax输出层)却依赖于@entity之间的不同来做分类。这在逻辑上存在一定的悖论。而丹琦的实现中,将文章中的不同实体按照出现顺序依次标号(relabel),对于模型性能的提升有较大帮助,但这一点其实并不应该被利用。

在Attensum Reader中,虽然模型逻辑上没有悖论,但是实际操作中也是将不同的@entity使用不同的embedding,间接的赋予@entity编号语义。为什么不遵从数据集原意,而将所有的@entity设置同一个embedding,而用mask标识出同一个@entity标号在文中哪些位置出现?

\section{动机}
Reader添加特征的实验一直没有进展,而多项实验从不同侧面暗示上述数据处理方式存在问题,为了验证数据处理方式的问题,遵从数据集合原意测试上述Reader的真实性能。即,测试@entity的编号以及在NN中的表示对于Reader的影响,毕竟这两者不应成为模型性能的来源。
 
\section{需要做的实验}
\subsection{第一优先级的实验}
\begin{enumerate}
    \item  {\bf  乐高代码迁移}:将模型转移到乐高上去,期待从根本上提速;
    \item  {\bf  数据构造}:取出@entity编号的频率影响,使得各个@entity编号出现次数相近,这样减少不同@entity编号之间的差异;
    \item  {\bf  Attention Reader去除relabel操作下的性能实验}:验证relabel对于模型性能的影响;
    \item  {\bf  Attention Reader使用新构造的数据}:验证Attention Reader对于@entity编号的依赖程度;
    \item  {\bf  Attensum Reader中所有实体使用同一个embeding,使用mask标识出同一个实体出现的位置}:完全排除掉@entity之间的语义差异,最符合数据集原意;
\end{enumerate}

\subsection{第二优先级的实验以及准备}
\begin{enumerate}
    \item  {\bf  基于CNN数据集正在进行的实验}:等待在乐高上的模型运行成功;
    \item  {\bf  SVM}:SVM的实验继续运行,及时很慢也需要有个结果出来;
    \item  {\bf 切换数据集}:由于之前的核心贡献点并不同数据集强依赖,如果证实CNN数据集有问题,迅速切换是比较合适的选择;
\end{enumerate}
 
\end{document}
